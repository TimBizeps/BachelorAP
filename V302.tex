\documentclass[
  bibliography=totoc,     % Literatur im Inhaltsverzeichnis
  captions=tableheading,  % Tabellenüberschriften
  titlepage=firstiscover, % Titelseite ist Deckblatt
]{scrartcl}

% Paket float verbessern
\usepackage{scrhack}

% Warnung, falls nochmal kompiliert werden muss
\usepackage[aux]{rerunfilecheck}

% unverzichtbare Mathe-Befehle
\usepackage{amsmath}
% viele Mathe-Symbole
\usepackage{amssymb}
% Erweiterungen für amsmath
\usepackage{mathtools}

% Fonteinstellungen
\usepackage{fontspec}
% Latin Modern Fonts werden automatisch geladen
% Alternativ:
%\setromanfont{Libertinus Serif}
%\setsansfont{Libertinus Sans}
%\setmonofont{Libertinus Mono}
\recalctypearea % Wenn man andere Schriftarten gesetzt hat,
% sollte man das Seiten-Layout neu berechnen lassen

% deutsche Spracheinstellungen
\usepackage{polyglossia}
\setmainlanguage{german}


\usepackage[
  math-style=ISO,    % ┐
  bold-style=ISO,    % │
  sans-style=italic, % │ ISO-Standard folgen
  nabla=upright,     % │
  partial=upright,   % ┘
  warnings-off={           % ┐
    mathtools-colon,       % │ unnötige Warnungen ausschalten
    mathtools-overbracket, % │
},                       % ┘
]{unicode-math}

% traditionelle Fonts für Mathematik
\setmathfont{Latin Modern Math}
% Alternativ:
%\setmathfont{Libertinus Math}

\setmathfont{XITS Math}[range={scr, bfscr}]
\setmathfont{XITS Math}[range={cal, bfcal}, StylisticSet=1]

% Zahlen und Einheiten
\usepackage[
locale=DE,                   % deutsche Einstellungen
separate-uncertainty=true,   % immer Fehler mit \pm
per-mode=symbol-or-fraction, % / in inline math, fraction in display math
]{siunitx}

% chemische Formeln
\usepackage[
version=4,
math-greek=default, % ┐ mit unicode-math zusammenarbeiten
text-greek=default, % ┘
]{mhchem}

% richtige Anführungszeichen
\usepackage[autostyle]{csquotes}

% schöne Brüche im Text
\usepackage{xfrac}

% Standardplatzierung für Floats einstellen
\usepackage{float}
\floatplacement{figure}{htbp}
\floatplacement{table}{htbp}

% Floats innerhalb einer Section halten
\usepackage[
section, % Floats innerhalb der Section halten
below,   % unterhalb der Section aber auf der selben Seite ist ok
]{placeins}

% Seite drehen für breite Tabellen: landscape Umgebung
\usepackage{pdflscape}

% Captions schöner machen.
\usepackage[
  labelfont=bf,        % Tabelle x: Abbildung y: ist jetzt fett
  font=small,          % Schrift etwas kleiner als Dokument
  width=0.9\textwidth, % maximale Breite einer Caption schmaler
]{caption}
% subfigure, subtable, subref
\usepackage{subcaption}

% Grafiken können eingebunden werden
\usepackage{graphicx}
% größere Variation von Dateinamen möglich
\usepackage{grffile}

% schöne Tabellen
\usepackage{booktabs}

% Verbesserungen am Schriftbild
\usepackage{microtype}

% Literaturverzeichnis
\usepackage[style=alphabetic,]{biblatex}
% Quellendatenbank
\addbibresource{lit.bib}

% Hyperlinks im Dokument
\usepackage[
  unicode,        % Unicode in PDF-Attributen erlauben
  pdfusetitle,    % Titel, Autoren und Datum als PDF-Attribute
  pdfcreator={},  % ┐ PDF-Attribute säubern
  pdfproducer={}, % ┘
]{hyperref}
% erweiterte Bookmarks im PDF
\usepackage{bookmark}

% Trennung von Wörtern mit Strichen
\usepackage[shortcuts]{extdash}

\title{V302: Brückenschaltungen}
\author{
  Simon Schulte
  \texorpdfstring{
    \\
    \href{mailto:simon.schulte@udo.edu}{simon.schulte@udo.edu}
  }{}
  \texorpdfstring{\and}{, }
  Tim Sedlaczek
  \texorpdfstring{
    \\
    \href{mailto:tim.sedlaczek@udo.edu}{tim.sedlaczek@udo.edu}
  }{}
}
\publishers{TU Dortmund – Fakultät Physik}

\date{Durchführung: 20.12.2016\\
      Abgabe: ??.01.2016}


\begin{document}

\maketitle
\thispagestyle{empty}
\tableofcontents
\newpage
\section{Zielsetzung}
\label{sec:zielsetzung}
Ziel des Versuchs ist es, mit Hilfe verschiedener Brückenschaltungen Widerstände,
Kapazitäten und Induktivitäten zu bestimmen, sowie eine Brückenschaltung, die
mit Wechselspannung betrieben wird, auf ihre Frequenzanhängigkeit zu untersuchen.
\section{Theorie}
\label{sec:theorie}
\subsection{Die allgemeine Brückenschaltung}
\label{sec:allgemeinebrückenschaltung}
\begin{figure}[htb]
  \centering
  \includegraphics[width=0.5\textwidth]{V3021.png}
  \caption{Schaltplan der Allgemeinen Brückenschaltung}
  \label{fig:V3021}
\end{figure}
In Abbildung \ref{fig:V3021} zu sehen ist der Aufbau der allgemeinen
Brückenschaltung. An den Punkten A und B wird die Brückenspannung abgegriffen.
Sie wird durch das Verhältnis der eingebrachten Widerstände $R_1$ und $R_4$
bestimmt. Aus der Knotenregel
\begin{equation}
  \sum_{\mathup{k}} I_{\mathup{k}}=0
\end{equation}
\label{Knotenregel}
und der Maschenregel
\begin{equation}
  \sum_{\mathup{k}} U_{\mathup{k}}=0
\end{equation}
\label{Maschenregel}
folgen die Zusammenhänge:
\begin{align*}
    I_1 &= I_2 & U &= -R_1 I_1 + R_3 I_3\\
    I_3 &= I_4 & U &= -R_4 I_4 + R_2 I_2 \\
    \shortintertext{sowie}
    U_{\mathup{S}} &= I_1 (R_1 + R_2).
\end{align*}
Daraus ergibt sich die Brückenspannung $U$
\begin{equation}
    U = \frac{R_2 R_3 - R_1 R_4}{(R_3 + R_4)(R_1 + R_2)} U_{\mathup{S}}.
    \label{eq:brückenspannung}
\end{equation}
Bei den Messungen dieses Versuchs wird immer, bei dem Oszillographen, die
Nullspannung anvisiert. Dies ist der Fall, wenn die Abgleichbedingung
\begin{equation}
    R_1 R_4 = R_2 R_3
    \label{eq:abgleich}
\end{equation}
gegeben ist. Für komplexe Widerstände folgt:
\begin{equation}
    Z = X + \mathup{i}Y.
\end{equation}
mit dem Wirkwiderstand X und dem Blindwiderstand Y. Die komplexen
Wechselstromwiderstände von einer idealen Kapazität $C$ und einer idealen
Induktivität $L$ sind, in Abhängigkeit von der Frequenz $\omega$, folgendermaßen
definiert:
\begin{equation}
    Z_C = -\frac{\mathup{i}}{\omega C} \quad \mathup{und} \quad Z_L = \mathup{i} \omega L.
    \label{eq:impedanzen}
\end{equation}
Zwei komplexe Zahlen sind nur dann gleich, wenn die Real- und die Imaginärteile
dieser gleich sind. Für komplexe Widerstände ergibt sich aus Gleichung
\ref{eq:abgleich} die Abgleichbedingung
\begin{equation}
    \label{eq:abgleich_komplex}
    Z_1 Z_4 = Z_2 Z_3.
\end{equation}
Damit folgt
\begin{align}
    \label{eq:abgleich_komplex_1}
    X_1 X_4 - Y_1 Y_4 &= X_2 X_3 - Y_2 Y_3 \\
    \shortintertext{und}
    \label{eq:abgleich_komplex_2}
    X_1 Y_4 + X_4 Y_1 &= X_2 Y_3 + X_3 Y_2.
\end{align}
\newpage
\subsection{Die Wheatstonesche Brücke}
\begin{figure}[htb]
  \centering
  \includegraphics[width=0.5\textwidth]{V3022.png}
  \caption{Schaltplan der Wheatstoneschen Brücke}
  \label{fig:V3022}
\end{figure}
In Abbildung \ref{fig:V3022} zu sehen ist die Wheatstonesche Brückenschaltung.
Sie wird genutzt, um einen ohmschen Widerstand $R_x$ zu bestimmen, der unbekannt
ist. Bei einer Wheatstoneschen Brückenschaltung kann mit Gleich- und
Wechselstrom gearbeitet werden. Durch die Abgleichbeziehung ergibt sich,
dass, durch die Änderung des Verhältnisses von $R_3$ und $R_4$, die
Brückenspannung auf 0 gebracht wird. Damit folgt
\begin{equation}
    \label{eq:wheatstone}
    R_{\mathup{x}} = R_2 \frac{R_3}{R_4}
\end{equation}
\newpage
\subsection{Die Kapazitätsmessbrücke}
\begin{figure}[htb]
  \centering
  \includegraphics[width=0.5\textwidth]{V3023.png}
  \caption{Schaltplan der Kapazitätsmessbrücke}
  \label{fig:V3023}
\end{figure}
In Abbildung \ref{fig:V3023} zu sehen ist die Brückenschaltung für die
Kapazitätsmessbrücke. Nun wird der unbekannte Widerstand $R_x$ durch einen
Kondensator $C_x$ ersetzt. Da der Kondensator allerdings auch verlustbehaftet
ist, ist im Schaltbild \ref{fig:V3023} ein fiktiver Widerstand $R_x$
berücksichtigt. Aus den beiden vorherigen Ausgleichsbedingungen folgt für
die Kapazitätsmessbrücke der Zusammenhang
\begin{align}
    \label{eq:kapazitätsmessbrücke_R}
    R_{\mathup{x}} &= R_2 \frac{R_3}{R_4} \\
    \shortintertext{und}
    \label{eq:kapazitätsmessbrücke_C}
    C_{\mathup{x}} &= C_2 \frac{R_4}{R_3}.
\end{align}

\newpage
\subsection{Die Induktivitätsmessbrücke}
\begin{figure}[htb]
  \centering
  \includegraphics[width=0.5\textwidth]{V3024.png}
  \caption{Schaltplan der Induktivitätsmessbrücke}
  \label{fig:V3024}
\end{figure}
In Abbildung \ref{fig:V3024} zu sehen ist die Brückenschaltung für die
Induktivitätsmessbrücke. Induktivitäten werden dabei analog wie bei der
Kapazitätsmessbrücke bestimmt, mit dem Unterschied, dass die Kapazitäten
durch Induktivitäten ersetzt werden. Dabei folgen aus den Abgleichbedingungen
folgende Zusammenhänge
\begin{align}
    \label{eq:induktivitätsmessbrücke_R}
    R_{\mathup{x}} &= R_2 \frac{R_3}{R_4} \\
    \shortintertext{und}
    \label{eq:induktivitätsmessbrücke_L}
    L_{\mathup{x}} &= L_2 \frac{R_3}{R_4}.
\end{align}
\newpage
\subsection{Die Maxwell-Brücke}
\begin{figure}[htb]
  \centering
  \includegraphics[width=0.5\textwidth]{V3025.png}
  \caption{Schaltplan der Maxwell-Brücke}
  \label{fig:V3025}
\end{figure}
In Abbildung \ref{fig:V3025} zu sehen ist die Brückenschaltung für die Maxwell-
Brücke. Hier wird auf die Induktivität $L_2$, welche bei der Induktivitätsmessbrücke
noch verwendet wurde, verzichtet. Stattdessen wird mit einem Kondensator $C_4$
gearbeitet. Aus den Abgleichbedingungen folgen dann die Beziehungen
\begin{align}
    \label{eq:maxwell_R}
    R_{\mathup{x}} &= \frac{R_2 R_3}{R_4} \\
    \shortintertext{und}
    \label{eq:maxwell_L}
    L_{\mathup{x}} &= R_2 R_3 C_4.
\end{align}

\newpage
\subsection{Die Wien-Robinson-Brücke}
\begin{figure}[htb]
  \centering
  \includegraphics[width=0.5\textwidth]{V3026.png}
  \caption{Schaltplan der Wien-Robinson-Brücke}
  \label{fig:V3026}
\end{figure}
In Abbildung \ref{fig:V3026} zu sehen ist die Brückenschaltung für die
Wien-Robinson-Brücke. Nun werden keine unbekannten Induktivitäten, Widerstände
oder Kapazitäten mehr genutzt, sondern es wird die Frequenzabhängigkeit
untersucht. Die Wien-Robinson-Brücke fungiert als Sperrfilter. Ein Sperrfilter
foltert eine bestimmte Frequenz einer Spannungsquelle vollständig aus dem
Spektrum heraus. Das Verhältnis zwischen Brückenspannung $U$ und Quellspannung
$U_Q$ kann man mit Hilfe der Kirchhoffschen Regeln \ref{Knotenregel} als
\begin{equation}
    \label{eq:wien_robinson_1}
    \left|\frac{U}{U_{\mathup{S}}}\right|^2 = \frac{\left(\omega^2 R^2 C^2 - 1\right)^2}{9\left(\left(1 - \omega^2 R^2 C^2\right)^2 + 9 \omega^2 R^2 C^2\right)}.
\end{equation}

ausdrücken.
\newpage
\section{Durchführung}
\label{sec:durchführung}
\subsection{Widerstandsmessung}
Als erstes wird in dem Versuch mit einer Wheatstoneschen Brücke, wie in
Abbildung \ref{fig:V3022} dargestellt, gearbeitet. Sie ist an einer
Wechselstromspannungsquelle angeschlossen. Zu bestimmen ist der Widerstand $R_x$.
Dieser ist unbekannt. $R_x$ wird in Reihe zum bekannten Widerstand $R_2$
geschaltet. Außerdem wird ein variabler Widerstand $R_3$ bzw. $R_4$ parallel,
in Form eines Potentiometers, bis \SI{1000}{\kilo\ohm}, verschaltet. Zwischen
$R_x$ und $R_2$ wird der erste Abgriff des Oszillographen platziert. Der zweite
Abgriff des Oszillographen greift zwischen den beiden Widerständen des
Potentiometers ab. Um nun $R_x$ zu bestimmen, wird das am Potentiometer
regelbare Verhältnis der beiden Widerstände $R_3$ und $R_4$ so eingestellt, dass
der Oszillograph eine Nullspaannung anzeigt. Diese Messung wird an zwei
verschiedenen unbekannten Widerständen $R_x$ durchgeführt, mit drei verschiedenen
$R_2$-Widerständen, um die Messgenauigkeit zu verbessern.
\subsection{Kapazitätsmessung}
Als nächstes wird eine Kapazitätsmessung durchgeführt. Dafür verändert man die
Schaltung, wie in \ref{fig:V3023} dargestellt. $R_2$ wird, unter der Annahme,
dass der zu bestimmende Kondensator $C_x$ ideal ist und keinen Wirkwiderstand
$R_x$ besitzt, vorerst nicht in die Schaltung eingebracht. Wie auch schon bei der
Widerstandsmessung wird erneut die Nullspannung gesucht. Diese Messung wird
für einen weiteren idealen Kondensator wiederholt, dabei wird $C_2$ für eine
bessere Messgenauigkeit variiert. Bei der nächsten Messung wird ein Kondensator
verwendet, der einen Wirkanteil in seinem Widerstand hat. Dabei wird nun der
zweite variable Widerstand $R_2$ mit in die Schaltung eingebracht. Auch hier
wird eine Nullspannung gesucht. Dabei wird das Verhältnis von $R_3$ und $R_4$
des ersten Potentiometers verändert, bis die Spannung ein Minimum findet.
Anschließend wird dies mit dem Potentiometer $R_2$ ebenfalls durchgeführt.
Dabei wird so abwechselnd variiert, bis ein absolutes Minimum nahe Null gefunden
wird. Dabei werden beide eingestellten Widerstände notiert. Außerdem wird jede
Messung mit drei verschiedenen Kapazitäten $C_2$ durchgeführt.
\newpage
\subsection{Induktivitätsmessung}
Als nächstes wird eine Induktivitätsmessung durchgeführt. Dafür wird ein Aufbau.
wie in Abbildung \ref{fig:V3024} zu sehen, genutzt. Dieser unterscheidet sich
zu dem zuvor verwendeten nur darin, dass die Reihenschaltung aus zwei
Kondesatoren durch eine aus zwei Induktivitäten ersetzt wird. Diese Messung
ist, vom Ablauf her, analog zu der Kapazitätsmessung. Um hier eine bessere
Messgenauigkeit zu gewährleisten wird die Messung mit drei verschiedenen
$L_2$-Induktivitäten durchgeführt.
Daraufhin wird mit einer Maxwell-Brücke, wie in Abbildung \ref{fig:V3025} zu
sehen, dieselbe Induktivität, wie bei der ersten Induktivitätsmessung,
bestimmt. Dazu werden $R_3$ und $R_4$ erneut solange variiert, bis die mit dem
Oszillographen gemessene Spannungsamplitude Null beträgt. Die Werte werden dann
für $R_3$ und $R_4$ bestimmt und die Messung wird für zwei weitere $R_2$-Widerstände
wiederholt.

\subsection{Untersuchung der Frequenzanhängigkeit einer Wien-Robinson-Brücke}
Als letztes wird die Frequenzabhängigkeit einer Brückenschaltung untersucht.
Dafür nutzt man eine Wien-Robinson-Brücke, wie in Abbildung \ref{fig:V3026}
zu sehen. Der in der Schaltung enthaltene Wechselstromgenerator liefert eine
Quellspannung $U_Q$. Diese Spannung hat den Vorteil, dass sie eine variable bzw.
einstellbare Frequenz hat. Zur Messung wird dann die Brückenspannung $U$ mit dem
Oszillographen gegen die Frequenz $f$ zwischen \SI{20}{\hertz} und \SI{20}{\kilo\hertz}
aufgenommen. Es werden 37 Werte aufgenommen für die Frequenz, $U_Q$ und $U$.
Dabei wird auf das Spannungsminimum von $U$ besonders Wert gelegt, indem man
vorallem Frequenzen bestimmt, die nahe an dem Spannungsminimum von
\SI{0.0152}{\volt} bei einer Frequenz von \SI{378}{\hertz} liegen.
\newpage
\end{document}
