\documentclass[
  bibliography=totoc,     % Literatur im Inhaltsverzeichnis
  captions=tableheading,  % Tabellenüberschriften
  titlepage=firstiscover, % Titelseite ist Deckblatt
]{scrartcl}

% Paket float verbessern
\usepackage{scrhack}

% Warnung, falls nochmal kompiliert werden muss
\usepackage[aux]{rerunfilecheck}

% unverzichtbare Mathe-Befehle
\usepackage{amsmath}
% viele Mathe-Symbole
\usepackage{amssymb}
% Erweiterungen für amsmath
\usepackage{mathtools}

% Fonteinstellungen
\usepackage{fontspec}
% Latin Modern Fonts werden automatisch geladen
% Alternativ:
%\setromanfont{Libertinus Serif}
%\setsansfont{Libertinus Sans}
%\setmonofont{Libertinus Mono}
\recalctypearea % Wenn man andere Schriftarten gesetzt hat,
% sollte man das Seiten-Layout neu berechnen lassen

% deutsche Spracheinstellungen
\usepackage{polyglossia}
\setmainlanguage{german}


\usepackage[
  math-style=ISO,    % ┐
  bold-style=ISO,    % │
  sans-style=italic, % │ ISO-Standard folgen
  nabla=upright,     % │
  partial=upright,   % ┘
  warnings-off={           % ┐
    mathtools-colon,       % │ unnötige Warnungen ausschalten
    mathtools-overbracket, % │
},                       % ┘
]{unicode-math}

% traditionelle Fonts für Mathematik
\setmathfont{Latin Modern Math}
% Alternativ:
%\setmathfont{Libertinus Math}

\setmathfont{XITS Math}[range={scr, bfscr}]
\setmathfont{XITS Math}[range={cal, bfcal}, StylisticSet=1]

% Zahlen und Einheiten
\usepackage[
locale=DE,                   % deutsche Einstellungen
separate-uncertainty=true,   % immer Fehler mit \pm
per-mode=symbol-or-fraction, % / in inline math, fraction in display math
]{siunitx}

% chemische Formeln
\usepackage[
version=4,
math-greek=default, % ┐ mit unicode-math zusammenarbeiten
text-greek=default, % ┘
]{mhchem}

% richtige Anführungszeichen
\usepackage[autostyle]{csquotes}

% schöne Brüche im Text
\usepackage{xfrac}

% Standardplatzierung für Floats einstellen
\usepackage{float}
\floatplacement{figure}{htbp}
\floatplacement{table}{htbp}

% Floats innerhalb einer Section halten
\usepackage[
section, % Floats innerhalb der Section halten
below,   % unterhalb der Section aber auf der selben Seite ist ok
]{placeins}

% Seite drehen für breite Tabellen: landscape Umgebung
\usepackage{pdflscape}

% Captions schöner machen.
\usepackage[
  labelfont=bf,        % Tabelle x: Abbildung y: ist jetzt fett
  font=small,          % Schrift etwas kleiner als Dokument
  width=0.9\textwidth, % maximale Breite einer Caption schmaler
]{caption}
% subfigure, subtable, subref
\usepackage{subcaption}

% Grafiken können eingebunden werden
\usepackage{graphicx}
% größere Variation von Dateinamen möglich
\usepackage{grffile}

% schöne Tabellen
\usepackage{booktabs}

% Verbesserungen am Schriftbild
\usepackage{microtype}

% Literaturverzeichnis
\usepackage[style=alphabetic,]{biblatex}
% Quellendatenbank
\addbibresource{lit.bib}
\addbibresource{programme.bib}

% Hyperlinks im Dokument
\usepackage[
  unicode,        % Unicode in PDF-Attributen erlauben
  pdfusetitle,    % Titel, Autoren und Datum als PDF-Attribute
  pdfcreator={},  % ┐ PDF-Attribute säubern
  pdfproducer={}, % ┘
]{hyperref}
% erweiterte Bookmarks im PDF
\usepackage{bookmark}

% Trennung von Wörtern mit Strichen
\usepackage[shortcuts]{extdash}

\title{V21: Optisches Pumpen}
\author{
  Simon Schulte
  \texorpdfstring{
    \\
    \href{mailto:simon.schulte@udo.edu}{simon.schulte@udo.edu}
  }{}
  \texorpdfstring{\and}{, }
  Tim Sedlaczek
  \texorpdfstring{
    \\
    \href{mailto:tim.sedlaczek@udo.edu}{tim.sedlaczek@udo.edu}
  }{}
}
\publishers{TU Dortmund – Fakultät Physik}

\date{Durchführung: 09.05.2018\\
      Abgabe: 01.06.2018}

\begin{document}

\maketitle
\thispagestyle{empty}
\setcounter{page}{1}
\pagenumbering{arabic}
\section{Theorie}
\label{sec:theorie}
Ziel des Versuchs ist die Untersuchung von energetischen Übergängen zwischen den
Zeeman-Niveaus in Rubidium-Isotopen im Magnetfeld. Daraus lassen sich die
Lande-Faktoren und der Kernspin der Isotope berechnen. \\
In allen Atomen befinden sich Elektronen auf sogenannten Schalen um den
Atomkern. Auf diesen Schalen besitzen sie feste Energien, zwischen welchen
allerdings Übergänge stattfinden können. Dazu muss Energie, die exakt der
Differenz der Niveaus entspricht von den Elektronen abgegeben oder aufgenommen
werden. Dies geschieht über die Aufnahme oder Abgabe von Photonen der
passenden Frequenz. Die äußeren Niveaus, also jene mit den größten Energien,
sind dabei je nach Temperatur unterschiedlich besetzt. Die Besetzungszahl eines
Niveaus folgt dabei einer Boltzmann Verteilung, sodass für das Verhältnis der
Besetzungszahlen zweier Niveaus gilt:
%
\begin{equation}
  \frac{N_2}{N_1}=\frac{g_2}{g_1}\frac{\exp\left(-\sfrac{W_2}{\symup{k_B}T}\right)}{\exp(-\sfrac{W_1}{\symup{k_B}T})}
\end{equation}
%
Hierbei beschreiben die $W_i$ die Energien der Zustände, die $g_{\symup{i}}$ die
statistischen Gewichte, welche beschreiben, wie viele Zustände zu der entsprechenden
Energie gehören, sowie $\symup{k_B}$ die Boltzmann-Konstante und $T$ die
Temperatur. \\
Mit Hilfe des optischen Pumpens kann eine Abweichung von dieser Verteilung
erzeugt werden. Diese kann soweit gehen, dass eine Besetzungsinversion
entsteht. Eine so entstehende nicht-thermische Verteilung ermöglicht es,
einzelne Übergänge zu induzieren, sodass Photonen dieser Energie überwiegend
das Spektrum prägen. Über die Vermessung dieses Lichtes lässt sich der
Energieabstand $\symup{h}\nu=W_2-W_1$ der Niveaus messen.

\subsection{Energieniveaus der Elektronen eines Atomes}
%
Die Verteilung der Elektronen auf verschiedene Energieniveaus innerhalb des
Atomes folgt aus dem Paulischen Ausschließungsprinzip verschiedener Quantenzahlen. Aus dem
Gesamtdrehimpuls $\vec{\mathup{J}}$ der Elektronenhülle folgt ein magnetisches Moment
$\vec{\mu}{\mathup{J}}=-g{\symup{J}}\mu_{\symup{B}}\vec{\mathup{J}}$. Der Lande-Faktor $g_{\symup{J}}$
berücksichtigt dabei, dass das Gesamtmoment aus Spin $\mathup{S}$ und Bahndrehimpuls $\mathup{L}$
der Elektronen folgt. Es gilt
%
\begin{align*}
  |\vec{\mu}{\mathup{J}}|=&g{\symup{J}}\mu_{\symup{B}}\sqrt{\mathup{J}(\mathup{J}+1)},\;\text{wobei} \\
  |\vec{\mu}{\mathup{L}}|=&-\mu{\symup{B}}\sqrt{\mathup{L}(\mathup{L}+1)}\;\text{und} \\
  |\vec{\mu}{\mathup{S}}|=&-g{\symup{S}}\mu_{\symup{B}}\sqrt{\mathup{S}(\mathup{S}+1)}\;.
\end{align*}
%
Das Gesamtdrehmoment $\vec{\mu}_{\mathup{J}}$ der Elektronenhülle präzediert dabei um den
Gesamtdrehimpuls $\vec{\mathup{J}}$. Daher trägt stets nur der parallele Anteil dieses
Drehimpulses zum magnetischen Moment bei. $g_{\symup{J}}$ ist also abhängig von
$\mathup{J}$, $\mathup{S}$ und $\mathup{L}$. Es ergibt sich
%
\begin{equation}
  g_{\symup{J}}=\frac{\num{3.0023}\mathup{J}(\mathup{J}+1)+\num{1.0023}\left(\mathup{S}(\mathup{S}+1)-\mathup{L}(\mathup{L}+1)\right)}{2\mathup{J}(\mathup{J}+1)}
  \label{eq:gj}
\end{equation}
%
Wird ein äußeres Magnetfeld angelegt, so spaltet sich die aus $L$, $S$ und $J$
resultierende Feinstruktur der Energieniveaus weiter auf. Dies ist in Abbildung
\ref{fig:aufspaltung} zu sehen. Bei Anlegen eines äußeren Magnetfeldes wird also über den Zeeman-Effekt jedes
Energieniveau in $2\mathup{J}+1$ Unterniveaus aufgespalten. Es gilt
%
\begin{equation}
  U_{\text{mag}}=-\vec{\mu}_{\mathup{J}\cdot\vec{B}}
\end{equation}
%
Hieraus folgt eine Richtungsquantelung der Energie:
%
\begin{equation}
  U_{\text{mag}}=g_{\symup{J}}\mu_{\symup{B}}B\cdot M_{\mathup{J}},\quad M_{\mathup{J}\in[-\mathup{J},\mathup{J}]}
  \label{penis}
\end{equation}
%
Die Anzahl der so
entstehenden Niveaus ist abhängig von dem sich aus Gesamtdrehimpuls der
Elektronenhülle $\mathup{J}$ und dem Kernspin $\mathup{I}$ ergebenden
Gesamtdrehimpuls $\mathup{F}$ des Atoms. Dabei ist der Kernspin der
Gesamtdrehimpuls eines Atomkerns um seinen Schwerpunkt.
%
\begin{figure}[h]
  \centering
  \includegraphics[width=0.75\textwidth]{Aufspaltung.jpg}
  \caption{Aufspaltung der Energieniveaus innerhalb eines Atomes. Dargestellt
  sind die Feinstrukturaufspaltung durch den Elektronenspin, die
  Hyperfeinstrukturaufspaltung durch den Kernspin, sowie die Zeeman-Aufspaltung
  durch ein äußeres Magnetfeld. \cite{anleitung}}
  \label{fig:aufspaltung}
\end{figure}

\subsection{Optisches Pumpen}
%
Die beiden in diesem Versuch betrachteten Rubidium-Isotope haben beide einen
Kernspin $I \neq 0$. Das Verfahren des optischen Pumpens wird anhand eines Beispiel-Systems
betrachtet. Es handelt sich dabei zunächst vereinfacht um ein Alkali-Atom ohne Kernspin $\vec{I}$.
Der Grundzustand ist der Zustand $^2S_{\sfrac{1}{2}}$. Angeregte Zustände sind
$^2P_{\sfrac{1}{2}}$, sowie $^2P_{\sfrac{3}{2}}$.
%
\begin{figure}[htb]
  \centering
  \includegraphics[width=0.9\textwidth]{Beispiel.jpg}
  \caption{Aufspaltung der $S$- und $P$-Niveaus eines Beispiel-Atomes. Rechts:
  Zeeman-Aufspaltung des $^2P_{\sfrac{1}{2}}$- sowie des
  $^2S_{\sfrac{1}{2}}$-Niveaus. \cite{anleitung}}
  \label{fig:beispiel}
\end{figure}
\noindent
Die Energieniveaus, sowie deren Aufspaltung im Magnetfeld sind in
Abbildung~\ref{fig:beispiel} skizziert. Das bei den rechts skizzierten
Übergängen emittierte Licht besitzt spezielle Polarisationseigenschaften. Das
$\sigma^+$-Licht ist rechtszirkular polarisiert (also
$\vec{s}\nparallel\vec{k}$), während das $\sigma^-$-Licht linkszirkular
polarisiert ist (also $\vec{s}\parallel\vec{k}$). Beide Linien erscheinen nur
parallel zum Magnetfeld in dieser Polarisation und sonst linear polarisiert.
Die $\pi$-Linien hingegen sind linear polarisiert und werden nicht parallel zum
Magnetfeld abgestrahlt, da es sich hierbei um Dipolstrahlung handelt.
%
\begin{figure}[htb]
  \centering
  \includegraphics[width=\textwidth]{Versuchsaufbau.pdf}
  \caption{Schematische Darstellung des Versuchsaufbaus \cite{anleitung}.}
  \label{fig:aufbau}
\end{figure}
\noindent
Abbildung~\ref{fig:aufbau} zeigt den schematischen Aufbau der verwendeten
Apparatur. Eine Spektrallampe strahlt Licht in die Apparatur ein. Aus diesem
Licht wird mit Hilfe eines $\text{D}_1$-Filters und eines Polarisationsfilters
rechtszirkular polarisiertes $\text{D}_1$-Licht. Der einzige mögliche
Energieübergang innerhalb der Atome in der Dampfzelle ist somit derjenige von
$^2\symup{S}{\sfrac{1}{2}}$ (mit $M{\mathup{J}}=-\sfrac{1}{2}$) zu
$^2P_{\sfrac{1}{2}}$ (mit $M_{\mathup{J}}=+\sfrac{1}{2}$), sowie umgekehrt durch
spontane Emission. Beim rückwärtigen Prozess werden allerdings beide
Zeeman-Niveaus des $^2\symup{S}_{\sfrac{1}{2}}$-Niveaus besetzt, sodass im
Laufe der Zeit das Niveau $^2\symup{S}_{\sfrac{1}{2}}$ (mit
$M_{\mathup{J}}=+\sfrac{1}{2}$) angereichert und das
$^2\symup{S}{\sfrac{1}{2}}$-Niveau (mit $M{\mathup{J}}=-\sfrac{1}{2}$)
\enquote{leergepumpt} wird. Dies hat eine Umkehr der Besetzung im S-Niveau zur
Folge. Experimentell beobachten lässt sich diese Entwicklung über die
Transparenz des Dampfes für Licht, wie in Abbildung \ref{fig:transparenz} veranschaulicht.
%
\begin{figure}[htb]
  \centering
  \includegraphics[width=0.6\textwidth]{transparenz.pdf}
  \caption{Verlauf der Transparenz der Dampfzelle in Abhängigkeit der Zeit. Das $M_{\mathup{J}}=-\sfrac{1}{2}$-Niveau wird \enquote{leergepumpt}\cite{anleitung}.}
  \label{fig:transparenz}
\end{figure}
\noindent
Wird Licht (einer bestimmten Energie) von dem Dampf absorbiert, so heißt dies,
dass Elektronen im Gas angeregt werden und dabei die Photonen absorbieren. Dies
ist natürlich nur möglich, wenn sich Elektronen in dem unteren Energieniveau
dieser Anregungen befinden. Je weniger Elektronen sich in diesem Niveau
befinden, desto unwahrscheinlicher ist eine Absorbtion des Photons und desto
transparenter erscheint das Gas. \\
\\
Es treten zwei Prozesse zur Abregung, also zur Emission von Photonen aus den
Gasatomen, auf. Bei der spontanen Emission begibt sich ein Elektron auf
ein niedrigeres Energieniveau und gibt dabei ein Photon der Energiedifferenz
ab. Bei der induzierten Emission verursacht ein eingestrahltes Photon
den beschriebenen Prozess, sodass hierbei zwei Photonen emittiert werden. Diese
gleichen sich in Energie, Polarisation und Impulsvektor. Welcher dieser beiden
Prozesse überwiegt, hängt von der Energie der Photonen ab. Die
Übergangswahrscheinlichkeit für spontane Emission ist proportional zu $\nu^3$,
und damit bei großen Frequenzen ausschlaggebend. Bei kleineren Energien, wie
etwa denen bei Zeeman-Übergängen kommt es quasi nur zu induzierter Emission.
Dabei nimmt dann die Transparenz der Dampfdruckkurve ab. Dies wird auch Resonanzfall
genannt.
Dies kann also sehr gut zur Vermessung dieser Übergänge verwendet werden. \\
Die Energiedifferenzen zwischen den Zeeman-Niveaus ist abhängig von der Stärke
des angelegten Magnetfeldes:
%
\begin{equation}
  h\nu=g_{\symup{J}}\mu_{\symup{B}}B\symup{\Delta}M_{\mathup{J}}
  \label{eq:übergangsenergie}
\end{equation}
%
Durch Einschalten des Magnetfeldes und Entstehen der Zeeman-Aufspaltung beginnt
das oben beschriebene Pumpen durch spontane Emission bis eine
Besetzungsumkehrung eintritt. Wenn allerdings das Magnetfeld den Wert
%
\begin{equation}
  B_{\mathup{m}}=\frac{4\pi m_0}{e_0g_J}\nu
\end{equation}
%
annimmt setzt induzierte Emission ein, sodass die Besetzungsinversion wieder
zerfällt. Das ist nur mit einer RF-Spule möglich. Diese würde ein weiteres horizontales
Magnetfeld erzeugen. Die Transparenz wird also nachdem sie Ihrem Maximum zustrebte plötzlich
wieder kleiner, wenn der oben beschriebene Wert erreicht wird, Abbildung~\ref{fig:transparenz_b}.
Dabei ist zu sehen, dass es einen negativen Nullpeak gibt. Dieser kommt dadurch
zu Stande, dass dort kein Zeeman Aufspaltung stattfindet. Dadurch gibt es kein
optisches Pumpen.
\begin{figure}[htb]
  \centering
  \includegraphics[width=0.6\textwidth]{transparenzkurve.pdf}
  \caption{Verlauf der Transparenz der Dampfzelle in Abhängigkeit des äußeren Magnetfeldes \cite{anleitung}.}
  \label{fig:transparenz_b}
\end{figure}
%
\subsection{Der Einfluss des Kernspins auf das Rubidium-Atom}
Rubidium hat einen Kernspin, der verschieden von 0 ist. Dadurch ändert sich die
Formel \eqref{eq:gj} zu
\begin{equation}
  g_{F}\,=\,g_{J} \frac{F(F+1)+J(J+1)-I(I+1)}{2F(F+1)}
\end{equation}
und die Formel \eqref{penis} zu
\begin{equation}
  U_{HF}\,=\,g_{F}\mu_{B} B.
  \label{lol}
\end{equation}
\subsection{Der quadratische Zeeman-Effekt}

Die Aufspaltung der Zeeman-Niveaus verhält sich allerdings nicht für alle Magnetfeldstärken
wie in Gleichung \eqref{lol} erkennbar. Für große Magnetfelder werden höhere
Ordnungen in Abhängigkeit des Magnetfeldes relevant, sodass die Zeeman-Aufspaltung durch
%
\begin{equation}
  \mathup{U}{\mathup{HF}} = g{\mathup{F}}\mu_{\mathup{B}}B + g_{\mathup{F}}^2\mu_{\mathup{B}}^2B^2\frac{(1-2M_{\mathup{F}})}{\symup{\Delta} \mathup{E_{Hy}}} - ....
  \label{eq:quad_zeeman}
\end{equation}
%
für einen Übergang von $M_F$ nach $M_{F+1}$ beschrieben wird. Dabei ist $\symup{\Delta} \mathup{E_{Hy}}$ die Energiedifferenz der
Hyperfeinstrukturaufspaltung zwischen den Niveaus $\mathup{F}$ und $\mathup{F}+1$.
Hier werden nun neben den Wechselwirkungen der magnetischen Momente mit dem äußeren
Magnetfeld auch diese zwischen den magnetischen Momenten berücksichtigt.

\section{Durchführung}
\label{sec:durchführung}

\subsection{Der Versuchsaufbau}
Abbildung~\ref{fig:aufbau} zeigt den schematischen Aufbau der Versuchsapparatur.
Ausgangspunkt des untersuchten Lichtes ist eine Spektrallampe. Dabei handelt es
sich um eine Rubidiumdampflampe. Das Licht dieser Lampe wird über eine
Sammellinse parallelisiert. Ein D1-Filter selektiert aus diesem
Licht ein schmales Frequenzspektrum heraus. Dabei handelt es sich um die
D1-Linie des Rubidiums, welche eine Wellenlänge von etwa
$\lambda=\SI{794.8}{\nano\metre}$ besitzt. Durch einen Polarisationsfilter
wird das Licht linear polarisiert, um anschließend durch ein
$\sfrac{\lambda}{4}$-Plättchen zirkular polarisiert zu werden. \\
Das nun hinreichend präparierte Licht fällt anschließend auf eine Dampfzelle.
Diese ist mit Rubidiumgas gefüllt und elektrisch beheizbar. Die Temperaturerhöhung
sorgt dafür, dass die thermische Verteilung der Energie breiter ist, sowie zur
Regulierung des Dampfdruckes innerhalb der Zelle. Die Zelle befindet sich
innerhalb mehrerer Helmholtzspulenpaare. Ein Paar ist horizontal ausgerichtet
(Horizontalfeldspule), während das zweite Paar vertikal ausgerichtet ist. Dazu
geschaltet ist eine Modulationsfeldspule (Sweep-Spule), die vertikal ausgerichtet
ist. Alle Spulen lassen sich
einzeln über die Feldströme steuern. Zusätzlich zu den Helmholtzspulen besitzt
die Apparatur auch eine RF-Spule. Mit dieser lassen sich elektromagnetische Wellen
bestimmter Frequenz erzeugen, die für die induzierte Emission verwendet werden.
Nachdem das Licht die Dampfzellle passiert hat wird es durch eine Sammellinse
wieder fokussiert und anschließend von einem Lichtdetektor vermessen.
Dazu wird ein Si-Photoelement verwendet, welches das Licht in ein elektrisches
Signal wandelt. Dieses wird durch ein Oszilloskop eingelesen und dargestellt.
Zur Abschirmung von Außenlicht wird der gesamte Messaufbau während der Messung
mit einer schwarzen Stoffdecke abgedunkelt.

\subsection{Die Versuchsdurchführung}
Da es sich hierbei um einen optischen Aufbau mit mehreren optischen Elementen
handelt, muss die Apperatur zunächst so eingestellt werden, dass das vermessene
Licht eine maximale Intensität aufweist. Dazu wird der Polarisationsfilter
sowie das $\sfrac{\lambda}{4}$-Plättchen zunächst aus dem Strahlengang entfernt,
alle \enquote{Gain}-Knöpfe werden auf $1$ gestellt und setzt die Zeitkonstante auf
\SI{100}{\milli\seconds}.
Da die Magnetfeldstärken der verwendeten Spulen in der Größenordnung des
Erdmagnetfeldes liegen, muss dieses möglichst kompensiert werden. Dazu wird der
gesamte Versuchsaufbau entlang der optischen Achse in Nord-Süd-Richtung
ausgerichtet. Dies dient zur Kompensation der horizontalen Komponente des
Magnetfeldes. Die vertikale Komponente wird durch die Vertikalspule
ausgeglichen. Dazu wird der Spulenstrom solange variiert, bis der auf dem
Oszilloskop dargestellte Bereich des Nullpeaks eine minimale Breite aufweist.
Zur Verringerung des Einflusses äußeren Lichtes wird der Aufbau nun durch eine
schwarze Stoffdecke abgeschirmt.
\clearpage
\noindent
Zur Vermessung der Zeeman-Aufspaltung wird über die RF-Spule stufenweise ein äußeres
hochfrequentes Magnetfeld angelegt ($\SI{100}{\kilo\hertz}$ bis $\SI{1}{\mega\hertz}$).
Dieses wird über eine Sinusspannung an einem HF-Generator erzeugt. Ziel ist es, die in
Abbildung~\ref{fig:transparenz_b} dargestellten \enquote{dips} auf dem Oszilloskop
zu untersuchen. Dazu werden für verschiedene Frequenzen an der RF-Spule die
Magnetfeldstärke im Resonanzfall, welcher sich durch minimale Transparenz der
Dampfzelle erkennen lässt, gemessen.\\
\clearpage
\section{Auswertung}
\label{sec:auswertung}
\subsection{Fehlerrechnung}
  Für die Fehlerrechnung sowie den mathematischen Teil der Auswertung wird auf
  $\textsc{Python}$ zurückgegriffen:\\
  Regressionen sowie deren Fehler wurden durch die $\textsc{Numpy}$ \cite{numpy} Funktion
  $\textsc{curve-fit}$ durchgeführt. Grafiken wurden mit $\textsc{Matplotlib}$ \cite{matplotlib}
  erstellt.
  Fehlerfortpflanzung wird durch die Bibliothek
  $\textsc{Uncertainties}$ \cite{uncertainties} automatisiert.

\subsection{Messwerte}
Für die Potenziometerskalen sind die Umrechnungsfaktoren vorgegeben.
Bei der Sweepspule sowie der Vertikalfeldspule beträgt er $\SI{0.1}{\ampere}$ pro Umdrehung und bei
der Horizontalfeldspule sind es $\SI{0.3}{\ampere}$ pro Umdrehung.
\begin{table}[H]
  \centering
  \caption{Die Stromstärken an Sweep- und Horizontalfeldspule bei den beiden Peaks.}
  \label{tab:1}
  \begin{tabular}{c c c c c}
    \toprule
    & \multicolumn{2}{c}{Sweepspule} & \multicolumn{2}{c}{Horizontalfeld}\\
    f in \si{\kilo\hertz} & 1.Peak in $\si{\milli\ampere}$ & 2.Peak in $\si{\milli\ampere}$ & 1.Peak in $\si{\milli\ampere}$ & 2.Peak $\si{\milli\ampere}$ \\
    \midrule
    100  & 614 & 734 &   0 &   0 \\
    201  & 364 & 604 &  60 &  60 \\
    300  & 136 & 496 &  90 &  90 \\
    401  & 157 & 638 & 105 & 105 \\
    500  & 182 & 770 & 120 & 120 \\
    602  & 418 & 477 & 120 & 165 \\
    700  & 419 & 604 & 135 & 180 \\
    800  & 444 & 528 & 150 & 210 \\
    900  & 477 & 438 & 165 & 240 \\
    1000 & 486 & 577 & 180 & 255 \\
    \bottomrule
  \end{tabular}
\end{table}
Das B-Feld eines Helmholtzspulenpaares berechnet sich nach:
\begin{equation}
  B = \mu_0 \frac{8}{\sqrt{125}}\cdot \frac{N}{R} \cdot I
\end{equation}
Dabei ist $N$ die Anzahl an Windungen pro Spule, $R$ der mittlere Radius
und $I$ die Stromstärke. Die Daten der verwendeten Spulen sind in Tabelle \ref{tab:2}
gegeben. Für $\mu_0$ wird der bei Scipy \cite{scipy} hinterlegte Wert verwendet.
\begin{table}[H]
  \centering
  \caption{Daten der Spulen.}
  \label{tab:2}
  \begin{tabular}{c c c}
    \toprule
    Spule & $N$ & $R$ in $\si{\meter}$ \\
    \midrule
    Sweep       &  11 & 0,1639 \\
    Horizontal  & 154 & 0,1579 \\
    Vertikal    &  20 & 0,11735 \\
    \bottomrule
  \end{tabular}
\end{table}
\clearpage
\noindent
Damit lassen sich die gemessenen B-Felder bestimmen:
\begin{table}[H]
  \centering
  \caption{Die Messwerte umgerechnet in Strom bzw. B-Feldstärke.}
  \label{tab:3}
  \begin{tabular}{c c c c c c c}
    \toprule
    & \multicolumn{2}{c}{Sweepspule} & \multicolumn{2}{c}{Horizontalfeld} & \multicolumn{2}{c}{B-Feld}\\
    f in \si{\kilo\hertz} & 1.Peak in $\si{\milli\ampere}$ & 2.Peak in $\si{\milli\ampere}$ & 1.Peak in $\si{\milli\ampere}$ & 2.Peak $\si{\milli\ampere}$ & 1.Peak in $\si{\micro\tesla}$ & 2.Peak in $\si{\micro\tesla}$ \\
    \midrule
    100  & 614 & 734 &   0 &   0 &  37,1 &  44,3 \\
    201  & 364 & 604 &  60 &  60 &  74,6 &  89,1 \\
    300  & 136 & 496 &  90 &  90 &  87,1 & 108,9 \\
    401  & 157 & 638 & 105 & 105 & 101,6 & 130,6 \\
    500  & 182 & 770 & 120 & 120 & 116,2 & 151,7 \\
    602  & 418 & 477 & 120 & 165 & 130,5 & 173,5 \\
    700  & 419 & 604 & 135 & 180 & 143,7 & 194,3 \\
    800  & 444 & 528 & 150 & 210 & 158,3 & 216,0 \\
    900  & 477 & 438 & 165 & 240 & 173,5 & 236,9 \\
    1000 & 486 & 577 & 180 & 255 & 187,2 & 258,4 \\
    \bottomrule
  \end{tabular}
\end{table}
\subsection{Bestimmung des lokalen Erdmagnetfeldes}
Beim ersten Plotten und Fitten fällt auf, dass fast alle Werte auf einer Geraden liegen.
Die ersten Messwerte, bei denen die Stromstärke an der Horizontalfeldspule noch bei 0 liegt,
reißen jedoch stärker aus. Wir wissen, dass beim minimieren des Horizontalfeldes
die Skala am Potenziometer nicht genau 0 angezeigt hat, wir aber 0 notiert haben.
Im weiteren Verlauf haben wir uns dann immer exakt an die Skala gehalten, weshalb es
wahrscheinlich zu einem systematischen Fehler kam, der sich in einer konkreten Differenz
in den Messwerten darstellt. Dieser Umstand betrifft jedoch lediglich die bestimmung des
Erdmagnetfeldes, da für die weiteren Auswertungsaufgaben die Steigung von
Bedeutung ist. Wir schätzen den Fehler auf etwa $0,085$ auf der Potenziometerskala,
was einer Stromstärke von $\SI{25.5}{\milli\ampere}$ und einem Magnetfeld von $\approx\SI{22.4}{\micro\tesla}$ entspricht. Die Vermutung wird dadurch
unterstützt, dass die beiden ersten Werte nach der Korrektur sehr gut auf den
Geraden liegen. Es lässt sich allerdings noch nicht feststellen, ob der Fehler bei den
ersten beiden Werten liegt oder ob die Skala generell zu viel anzeigt.
Schließlich werden die Werte in Tabelle \ref{tab:4} linear an
\begin{equation}
  y = m \cdot x + b
\end{equation}
gefittet.
\begin{table}[H]
  \centering
  \caption{Für Abbildung \ref{fig:plot1} verwendete B-Felder.}
  \label{tab:4}
  \begin{tabular}{c c c c c}
    \toprule
    & \multicolumn{2}{c}{ursprüngliche Messung} & \multicolumn{2}{c}{Werte mit Korrektur}\\
    f in \si{\kilo\hertz} & 1.Peak in $\si{\micro\tesla}$ & 2.Peak in $\si{\micro\tesla}$ & 1.Peak in $\si{\micro\tesla}$ & 2.Peak in $\si{\micro\tesla}$ \\
    \midrule
    100  &  37,1 &  44,3 &  59,4 &  66,7 \\
    201  &  74,6 &  89,1 &  52,2 &  66,7 \\
    300  &  87,1 & 108,9 &  64,8 &  86,5 \\
    401  & 101,6 & 130,6 &  79,2 & 108,2 \\
    500  & 116,2 & 151,7 &  93,9 & 129,3 \\
    602  & 130,5 & 173,5 & 108,1 & 151,1 \\
    700  & 143,7 & 194,3 & 121,3 & 171,9 \\
    800  & 158,3 & 216,0 & 136,0 & 193,7 \\
    900  & 173,5 & 236,9 & 151,1 & 214,5 \\
    1000 & 187,2 & 258,4 & 164,8 & 236,1 \\
    \bottomrule
  \end{tabular}
\end{table}
\begin{figure}
  \centering
  \includegraphics[width=\textwidth]{Plot2.pdf}
  \caption{Messwerte und korrigierte Werte mit Fit.}
  \label{fig:plot1}
\end{figure}
\clearpage
\noindent
Für Fit 1 und Fit 2 wurde durch die korrigierten Werte bei $\SI{100}{\kilo\hertz}$
und die ursprünglichen Messwerte von $\num{200}$ bis $\SI{1000}{\kilo\hertz}$ gefittet.
Für Fit 3 und Fit 4 wurde durch die ursprünglichen Werte bei $\SI{100}{\kilo\hertz}$
und die korrigierten Werte bei $\num{200}$ bis $\SI{1000}{\kilo\hertz}$ gefittet.
Die Fitparameter stehen in Tabelle \ref{tab:5}.
\begin{table}[H]
  \centering
  \caption{Fitparameter.}
  \label{tab:5}
  \begin{tabular}{c c c}
    \toprule
    Fit & $m$ & $b$ \\
    \midrule
    1  & \SI{0.1418(6)}{\micro\tesla\per\kilo\per\hertz} & \SI{45.1(4)}{\micro\tesla} \\
    2  & \SI{0.2128(4)}{\micro\tesla\per\kilo\per\hertz} & \SI{45.5(3)}{\micro\tesla} \\
    3  & \SI{0.1418(6)}{\micro\tesla\per\kilo\per\hertz} & \SI{22.8(4)}{\micro\tesla} \\
    4  & \SI{0.2128(4)}{\micro\tesla\per\kilo\per\hertz} & \SI{23.1(3)}{\micro\tesla} \\
    \bottomrule
  \end{tabular}
\end{table}
\noindent
Der y-Achsenabschnitt $b$ entspricht der Horizontalkomponente des Erdmagnetfeldes.
Für die Vertikalkomponente wurde am Potenziometer der Vertikalfeldspule ein Wert
von $\num{2.24}$ abgelesen. Das entspricht einer Stromstärke von $\SI{224}{\milli\ampere}$.
Damit ergibt sich ein Magnetfeld von $\SI{34.3}{\micro\tesla}$.\\
Die Horizontalkomponente des Erdmagnetfeldes liegt in Dortmund/Deutschland knapp unter
$\SI{20}{\micro\tesla}$ und die Vertikalkomponente bei $\SI{45}{\micro\tesla}$ \cite{GFZPot}.

\subsection{Landé $g_\mathup{F}$ und Kernspin der Rubidium Isotope}
Für die Berechnung der Landé-Faktoren werden die ursprünglichen Messwerte verwendet
und die beiden Werte bei $\SI{100}{\kilo\hertz}$ vernachlässigt. Durch die restlichen
Werte wird wie zuvor linear gefittet. Die Fitergebnisse sind in Abbildung \ref{fig:plot2}
zu sehen und die Fitparameter stehen in \ref{tab:6}.
\begin{table}[H]
  \centering
  \caption{Fitparameter.}
  \label{tab:6}
  \begin{tabular}{c c c}
    \toprule
    Fit & $m$ & $b$ \\
    \midrule
    1  & \SI{0.1419(8)}{\micro\tesla\per\kilo\per\hertz} & \SI{45.1(5)}{\micro\tesla} \\
    2  & \SI{0.2127(5)}{\micro\tesla\per\kilo\per\hertz} & \SI{45.5(3)}{\micro\tesla} \\
    \bottomrule
  \end{tabular}
\end{table}
\noindent
Die Energie der von der RF-Spule erzeugten Photonen beträgt $h \cdot \mathup{f}$.
Wobei $h$ die Plank-Konstante ist. Für diese wird wieder der Wert von Scipy \cite{scipy}
verwendet. Durch Einsetzen, von $h \cdot \mathup{f}$ für $\mathup{U}_\mathup{HF}$
in Formel \ref{eqn:zeeman}, und Umformen ergibt sich:
\begin{equation}
  B = \frac{h}{g_\mathup{F} \mu_\mathup{B}} \cdot \mathup{f}
\end{equation}
Damit folgt:
\begin{equation}
  m = \frac{h}{g_\mathup{F} \mu_\mathup{B}}
\end{equation}
Und schließlich:
\begin{equation}
  g_\mathup{F} = \frac{h}{m \mu_\mathup{B}}
\end{equation}
\begin{figure}[H]
  \centering
  \includegraphics[width=\textwidth]{Plot1.pdf}
  \caption{Messwerte mit Fit.}
  \label{fig:plot2}
\end{figure}
\noindent
Für das bohrsche Magneton wir ebenfalls der Wert von Scipy \cite{scipyconst} verwendet.
Damit ergeben sich für die beiden Isotope folgende $g_\mathup{F}$:
\begin{align*}
  g_{\mathup{F}_1} &= \num{0.504(3)}\\
  g_{\mathup{F}_2} &= \num{0.3359(8)}
\end{align*}
\noindent
Da Rubidium zu den Alkali-Metallen gehört besitzt die Elektronenhülle lediglich
einen Gesamtspin von $\mathup{J} = \frac{1}{2}$. Dabei ist $\mathup{S} = \frac{1}{2}$
und $\mathup{L} = 0$. In Formel \ref{eq:gj} eingesetzt ergibt sich damit:
$g_\mathup{J} = \num{2.0023}$.\\
Wenn nun in Formel (\ref{eqn:gflande}) $\mathup{F}$ durch $\mathup{I}+\mathup{J}$
ersetzt wird (mit $\mathup{J} = \frac{1}{2}$) und anschließend die Formel auf $\mathup{I}$ umgeformt wird,
folgt durch Einsetzen von $g_\mathup{F}$ und $g_\mathup{J}$ der Kernspin $\mathup{I}$
der Isotope:
\begin{equation}
  \mathup{I} = \frac{\frac{g_\mathup{J}}{g_\mathup{F}}-1}{2}
\end{equation}
\begin{align*}
  \mathup{I}_1 &= \num{1.49(1)}\\
  \mathup{I}_2 &= \num{2.48(1)}
\end{align*}
Literaturwerte dafür sind \cite{rubwi}:
\begin{align*}
  \mathup{I}_{^{87}\mathup{Rb}} &= \frac{3}{2}\\
  \mathup{I}_{^{85}\mathup{Rb}} &= \frac{5}{2}
\end{align*}
Damit ist offenbar das untere Isotop in Abbildung \ref{fig:plot2} $^{87}\mathup{Rb}$
und das obere $^{85}\mathup{Rb}$.

\subsection{Abschätzung des quadratischen Zeeman-Effektes}
Für die Abschätzung des quadratischen Zeeman-Effektes werden der lineare und der
quadratische Teil in Formel (\ref{eq:quad_zeeman}) einzeln für die bestimmten Werte
von $g_\mathup{F}$ und für die sich, aus den Parametern des 1.Fits (für $\SI{100}{\kilo\hertz}$)
und aus den Parametern des 2.Fits (für $\SI{1000}{\kilo\hertz}$) ergebenden B-Feldern berechnet.
Dabei wirken die beiden Werte der B-Felder als obere und untere Grenze.
Zusätzlich wird für $^{87}\mathup{Rb}$ $M_\mathup{F}=2$ verwendet, was sich aus
dem Kernspin und dem Spin des Elektrons ergibt und für $\mathup{F}=2$ dem maximal
möglichen Energieniveau entspricht. Genau so wird für $^{85}\mathup{Rb}$
$M_\mathup{F}=3$ verwendet. Die Energiedifferenz der Hyperfeinstruktur
$\symup{\Delta} \mathup{E_{Hy}}$ ist in der Anleitung mit
$\symup{\Delta} \mathup{E_{Hy}} = \SI{4.53e-24}{\joule}$ für $^{87}\mathup{Rb}$
und $\symup{\Delta} \mathup{E_{Hy}} = \SI{2.01e-24}{\joule}$ für $^{85}\mathup{Rb}$
gegeben. Die Ergebnisse stehen in Tabelle \ref{tab:7}.
\begin{table}[H]
  \centering
  \caption{Linearer und quadratischer Zeeman-Effekt.}
  \label{tab:7}
  \begin{tabular}{c c c}
    \toprule
    & $^{87}\mathup{Rb}$ & $^{85}\mathup{Rb}$ \\
    \midrule
    lin (\SI{59.3}{\micro\tesla}) & \SI{2.77(2)e-28}{\joule} & \SI{1.847(4)e-28}{\joule} \\
    lin (\SI{258.3}{\micro\tesla}) & \SI{12.06(7)e-28}{\joule} & \SI{8.04(2)e-28}{\joule} \\
    quad (\SI{59.3}{\micro\tesla}) & \SI{-5.08(6)e-32}{\joule} & \SI{-8.48(4)e-32}{\joule} \\
    quad (\SI{258.3}{\micro\tesla}) & \SI{-9.6(1)e-31}{\joule} & \SI{-1.610(8)e-30}{\joule} \\
    \bottomrule
  \end{tabular}
\end{table}
Der Einfluss des quadratischen Zeeman-Effektes liegt also maximal in einer
Größenordnung von \SI{e-30}{\joule}.

\subsection{Ausdruck des Signalbildes und Bestimmung des Isotopenverhältnisses}
Auf dem Ausdruck wird zunächst per Lineal die Tiefe der beiden Transparenzminima
bestimmt. Sie ist proportional zu der in der Probe vorhandenen Menge des jeweiligen
Isotops. Somit wird der jeweilige Anteil des Isotops an der Summe der beiden
Isotope berechnet.
\begin{table}[H]
  \centering
  \caption{Größe der Minima und prozentuale Anteile der Isotope.}
  \label{tab:8}
  \begin{tabular}{c c c}
    \toprule
    & $^{87}\mathup{Rb}$ & $^{85}\mathup{Rb}$ \\
    \midrule
    Minimum & \SI{2.4}{\centi\meter} & \SI{5.1}{\centi\meter} \\
    Anteil in \si{\percent} & \num{32} & \num{68} \\
    Naturwert \cite{seilnacht} \cite{rubwi} & \num{27.83} & \num{72.17} \\
    Abweichung in \si{\percent} & \num{15} & \num{-6} \\
    \bottomrule
  \end{tabular}
\end{table}

\section{Diskussion}
\label{sec:diskussion}
Insgesamt liefert der Versuch, abgesehen von der Messung des Erdmagnetfeldes, gute
Ergebnisse. Auch nach der Korrektur des systematischen Fehlers bei der Berechnung
des Erdmagnetfeldes liegt der Wert für die Horizontalkomponente mit
$\SI{22.8(4)}{\micro\tesla}$ bzw. $\SI{23.1(3)}{\micro\tesla}$ noch
$\approx \SI{3}{\micro\tesla}$ über dem Soll. Die berechnete Vertikalkomponente
mit $\SI{34.3}{\micro\tesla}$ sogar $\approx \SI{11}{\micro\tesla}$ unter dem
Literaturwert. Somit liegen noch weitere systematische Fehler vor. Möglicherweise
besitzt die Vertikalfeldspule ebenfalls einen Fehler im Potenziometer. Zusätzlich
war das Einstellen des Feldes nur mit begrenzter Genauigkeit möglich, da sich
die optimale Einstellung nur schwierig über die breite des Minimums bestimmen
ließ. Eine weitere mögliche Fehlerquelle ist, dass der Tisch möglicherweise
nicht ganz waagerechet ausgerichtet war und dadurch dann die Vertikalkomponente
etwas schwächer und die Horizontalkomponente etwas stärker wahrgenommen wurden.\\
Der ermittelte Kernspin ist bei beiden Isotopen mit $\num{1.49(1)}$ und $\num{2.48(1)}$
sehr dicht an dem Literaturwert.\\
Bei der Abschätzung des quadratischen Zeeman-Effektes ist deutlich zu erkennen,
dass dieser hier nur eine kleine Rolle spielt, da die Aufspaltung bei den von
uns betrachteten Magnetfeldern zwei bis vier Größenordnungen kleiner ist.\\
Das Isotopenverhältnis in der Probe ist nah an dem natürlichen Verhältnis.
Die Abmessung ist hier allerdings auch mit einer gewissen Ungenauigkeit verbunden,
da die Strichdicke in der Aufnahme einen Ablesefehler nicht unwahrscheinlich macht.
Tendenziell scheint die Probe aber leicht mit $^{87}\mathup{Rb}$ angereichert zu sein.
\newpage
\nocite{*}
\printbibliography

\end{document}
