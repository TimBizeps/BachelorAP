\documentclass[
  bibliography=totoc,     % Literatur im Inhaltsverzeichnis
  captions=tableheading,  % Tabellenüberschriften
  titlepage=firstiscover, % Titelseite ist Deckblatt
]{scrartcl}

% Paket float verbessern
\usepackage{scrhack}

% Warnung, falls nochmal kompiliert werden muss
\usepackage[aux]{rerunfilecheck}

% unverzichtbare Mathe-Befehle
\usepackage{amsmath}
% viele Mathe-Symbole
\usepackage{amssymb}
% Erweiterungen für amsmath
\usepackage{mathtools}

% Fonteinstellungen
\usepackage{fontspec}
% Latin Modern Fonts werden automatisch geladen
% Alternativ:
%\setromanfont{Libertinus Serif}
%\setsansfont{Libertinus Sans}
%\setmonofont{Libertinus Mono}
\recalctypearea % Wenn man andere Schriftarten gesetzt hat,
% sollte man das Seiten-Layout neu berechnen lassen

% deutsche Spracheinstellungen
\usepackage{polyglossia}
\setmainlanguage{german}


\usepackage[
  math-style=ISO,    % ┐
  bold-style=ISO,    % │
  sans-style=italic, % │ ISO-Standard folgen
  nabla=upright,     % │
  partial=upright,   % ┘
  warnings-off={           % ┐
    mathtools-colon,       % │ unnötige Warnungen ausschalten
    mathtools-overbracket, % │
},                       % ┘
]{unicode-math}

% traditionelle Fonts für Mathematik
\setmathfont{Latin Modern Math}
% Alternativ:
%\setmathfont{Libertinus Math}

\setmathfont{XITS Math}[range={scr, bfscr}]
\setmathfont{XITS Math}[range={cal, bfcal}, StylisticSet=1]

% Zahlen und Einheiten
\usepackage[
locale=DE,                   % deutsche Einstellungen
separate-uncertainty=true,   % immer Fehler mit \pm
per-mode=symbol-or-fraction, % / in inline math, fraction in display math
]{siunitx}

% chemische Formeln
\usepackage[
version=4,
math-greek=default, % ┐ mit unicode-math zusammenarbeiten
text-greek=default, % ┘
]{mhchem}

% richtige Anführungszeichen
\usepackage[autostyle]{csquotes}

% schöne Brüche im Text
\usepackage{xfrac}

% Standardplatzierung für Floats einstellen
\usepackage{float}
\floatplacement{figure}{htbp}
\floatplacement{table}{htbp}

% Floats innerhalb einer Section halten
\usepackage[
section, % Floats innerhalb der Section halten
below,   % unterhalb der Section aber auf der selben Seite ist ok
]{placeins}

% Seite drehen für breite Tabellen: landscape Umgebung
\usepackage{pdflscape}

% Captions schöner machen.
\usepackage[
  labelfont=bf,        % Tabelle x: Abbildung y: ist jetzt fett
  font=small,          % Schrift etwas kleiner als Dokument
  width=0.9\textwidth, % maximale Breite einer Caption schmaler
]{caption}
% subfigure, subtable, subref
\usepackage{subcaption}

% Grafiken können eingebunden werden
\usepackage{graphicx}
% größere Variation von Dateinamen möglich
\usepackage{grffile}

% schöne Tabellen
\usepackage{booktabs}

% Verbesserungen am Schriftbild
\usepackage{microtype}

% Literaturverzeichnis
\usepackage[style=alphabetic,]{biblatex}
% Quellendatenbank
\addbibresource{lit.bib}

% Hyperlinks im Dokument
\usepackage[
  unicode,        % Unicode in PDF-Attributen erlauben
  pdfusetitle,    % Titel, Autoren und Datum als PDF-Attribute
  pdfcreator={},  % ┐ PDF-Attribute säubern
  pdfproducer={}, % ┘
]{hyperref}
% erweiterte Bookmarks im PDF
\usepackage{bookmark}

% Trennung von Wörtern mit Strichen
\usepackage[shortcuts]{extdash}

\title{V201: Das Dulong-Petitsche Gesetz}
\author{
  Simon Schulte
  \texorpdfstring{
    \\
    \href{mailto:simon.schulte@udo.edu}{simon.schulte@udo.edu}
  }{}
  \texorpdfstring{\and}{, }
  Tim Sedlaczek
  \texorpdfstring{
    \\
    \href{mailto:tim.sedlaczek@udo.edu}{tim.sedlaczek@udo.edu}
  }{}
}
\publishers{TU Dortmund – Fakultät Physik}

\date{Durchführung: 22.11.2016\\
      Abgabe: 29.11.2016}


\begin{document}

\maketitle
\thispagestyle{empty}
\tableofcontents
\newpage
\section{Zielsetzung}
\label{sec:zielsetzung}
In diesem Versuch soll die Molwärme bei konstantem Druck von Blei, Graphit und
Aluminium durch eine Mischungskalometrie bestimmt werden. Mit diesen Ergebnissen
soll dann überprüft werden, ob die oszillatorische Bewegung der Atome
(oder Moleküle) eines Festkörpers noch mit klassischen Methoden beschrieben
werden kann oder ob eine korrekte Darstellung nur auf der Grundlage der
Quantenmechanik möglich ist.
\section{Theorie}
\label{sec:theorie}
Das Dulong-Petit-Gesetz besagt, dass die molare Wärmekapazität eines Festkörpers
den konstanten Wert, nämlich das Dreifache der universellen Gaskonstante $R$
\begin{equation}
  C=3R \approx 3 \cdot \SI{8.314}{\joule\per\mol\per\kelvin} \approx
  \SI{24.942}{\joule\per\mol\per\kelvin}
  \label{eqn:c-dulong}
\end{equation}
habe. Im Fokus dieses Experiments steht die Bestimmung der spezifischen
Wärmekapazität von Feststoffen. Die spezifische Wärmekapazität ist eine
stoffcharakteristische Eigenschaft, die Auskunft darüber gibt, inwiefern
zugeführte Energiemenge $\Delta Q$ und Temperaturanstieg $\Delta T$ an einem Körper im
Zusammenhang stehen. Mit $m$, für die Masse des Stoffes und c, der spezifischen
Wärmekapazität des Stoffes ergibt sich
\begin{equation}
  \Delta Q=mc\Delta T,
  \label{eqn:wärmemenge}
\end{equation}
mit der Einheit $\si{\joule\per\kilogram\per\kelvin}$
für die spezifische Wärmekapazität.
Man kann die Wärmekapazität allerdings nicht
nur auf die Masse beziehen, sondern mit dem Zusammenhang
\begin{equation}
  C=M \cdot c.
  \label{eqn:C}
\end{equation}
auch auf die Molwärme $C$. Diese misst man bei konstantem Volumen $C_V$ oder bei
konstantem Druck $C_P$.
Somit folgt für die Molwärme bei konstantem Volumen
\begin{equation}
  C_{\mathup{V}}=3R
  \label{eqn:Cv-dulong}
\end{equation}
Dabei ergibt sich die Allgemeine Gaskonstante $R$ aus der Lonschmidtschen Zahl $N_L$
und der Boltzmann-Konstante $k_B$.\\
Die Quantenmechanik liefert ein exakteres Modell. Sie geht davon aus, dass
Energie und somit Wärme nur gequantelt von Festkörpern aufgenommen oder
abgegeben werden kann. Für die innere Energie von einem Mol eines Stoffes folgt
in der Quantenmechanik die Beziehung
\begin{equation}
  \langle U_{\mathup{qu}}\rangle=\frac{3N_{\mathup{L}}\hbar\omega}{\mathup{exp}\left(\hbar\frac{\omega}{kT}\right)-1}.
  \label{eqn:Uqu}
\end{equation}
mit dem reduzierten Plankschen Wirkungsquantum $\hbar$ und der Schwingungsfrequenz der
Atome im Feststoff $\omega$. Im Modell der Quantenmechanik ist das Dulong-Petitsche
Gesetz nur ein Extremfall für hohe Temperaturen, da es sich erst bei hohen
Temperaturen dem klassischen Wert
\begin{equation}
  \langle U_{\mathup{kl}}\rangle=3RT
  \label{eqn:Ukl}
\end{equation}
annähert. Im Experiment ist es bedeutend einfacher die spezifische Wärmekapazität $c_{\mathup{k}}$ einer Probe bei
konstantem Druck zu bestimmen. $c_{\mathup{k}}$ lässt sich durch die Formel

\begin{equation}
  c_{\mathup{k}}=\frac{\left(c_{\mathup{w}}m_{\mathup{w}}+c_{\mathup{g}}m_{\mathup{g}}\right)\left(T_{\mathup{m}}-T_{\mathup{w}}\right)}{m_{\mathup{k}}\left(T_{\mathup{k}}-T_{\mathup{m}}\right)}
  \label{eqn:ck}
\end{equation}
bestimmen, die aus einer Betrachtung der im System erhaltenen Gesamtwärmemenge resultiert.
$c_{\mathup{w}}$ ist hierbei die spezifische Wärmekapazität von Wasser,
die mit $c_{\mathup{w}}=\SI{4.18}{\joule\per\gram\per\kelvin}$ vorgegeben ist.
Die Größen $m_{\mathup{w}}$, $m_{\mathup{k}}$, $T_{\mathup{w}}$, $T_{\mathup{m}}$ und $T_{\mathup{k}}$ müssen
experimentell bestimmt werden.
Desweiteren muss die spezifische Wärmekapazität des Kalorimeters $c_{\mathup{g}}m_{\mathup{g}}$ in einer
separaten Messung bestimmt werden. Derselbe wärmeenergieerhaltende Ansatz wie bei Formel (7) liefert

\begin{equation}
  c_{\mathup{g}}m_{\mathup{g}}=\frac{c_{\mathup{w}}m_{\mathup{y}}\left(T_{\mathup{y}}-T'_{\mathup{m}}\right)-c_{\mathup{w}}m_{\mathup{x}}\left(T'_{\mathup{m}}-T_{\mathup{x}}\right)}{\left(T'_{\mathup{m}}-T_{\mathup{x}}\right)},
  \label{eqn:cgmg}
\end{equation}
wobei auch hier $m_{\mathup{x}}$, $m_{\mathup{y}}$, $T_{\mathup{x}}$, $T_{\mathup{y}}$ und $T'_{\mathup{m}}$
experimentell zu bestimmen sind.
\clearpage

\section{Durchführung}
\label{sec:durchführung}
\subsection{Versuchsaufbau}
Um die spezifische Wärmekapazität $c_k$ eines Körpers zu ermitteln nutzt man ein
Mischungskalorimeter. Die röhrenförmigen Probekörper
Blei, Aluminium und Graphit werden in einem Wasserbad, welches von einer
Herdplatte erwärmt wird auf eine Temperatur von \SI{100}{\celsius} gebracht.
Danach werden die Probekörper jeweils in ein Dewa-Gefäß eingetaucht. Dieses ist
mit einer bestimmten Masse $m_w$ an Wasser mit einer Temperatur $T_w$ befüllt.
Das Wasser wird immer wieder ausgewechselt nachdem es einmal benutzt wurde.
Die Temperatur wird mit einem Thermoelement gemessen, welches mit der Probe
selbst und Eiswasser verbunden ist. Dieses ist \SI{0}{\celsius} kalt.
\subsection{Versuchsdurchführung}
Zunächst ist die spezifische Wärmekapazität $c_{\mathup{g}}m_{\mathup{g}}$ des
Dewar-Gefäßes zu bestimmen.
Dazu wird dieses mit ca. 600ml Wasser fast gänzlich gefüllt. Nach Bestimmung der
Masse des Wassers in einem Becherglas gießt man dieses in das Dewar-Gefäß und
bestimmt die Leermasse des Becherglases. Die Temperaturen von Wasser
$T_{\mathup{w}}$ und Dewar-Gefäß gleichen sich innerhalb einer kurzen Zeit an.
Zur Bestimmung der Wärmekapazität gießt man etwa die Hälfte dieses Wassers in
das Becherglas zurück, bestimmt die Menge $m_{\mathup{y}}$ durch Wiegen und
erwärmt dieses mit Hilfe einer Herdplatte auf etwa \SI{100}{\celsius}.
Wenn die erforderliche Temperatur im Becherglas erreicht ist, wird die
Temperatur $T_{\mathup{w}}$ des Wassers im Dewar-Gefäß bestimmt. Stellt sich nach
wenigen Minuten eine nur noch geringen Schwankungen unterliegende stationäre
Temperatur ein, so ist die Mischtemperatur $T_{\mathup{m}}$ für dieses System
erreicht.
Zur Bestimmung der Messgrößen für die spezifischen Wärmekapazitäten der
Probenstoffe geht man analog vor. Eine vorher im gleichen Becherglas gewogene
Wassermenge von etwa \SI{600}{\gram} wird in das selbe Dewar-Gefäß gefüllt. Die
holhzylinderförmigen Proben befinden sich, über einen Faden verbunden, an einem
kreisrunden Deckel. Mit Hilfe dieses Deckels kann die Probe sowohl über den Rand
eines Becherglases, als auch über den Rand des Dewar-Gefäßes in Wasser gehängt
werden. Zur Bestimmung des Gewichts wird zunächst die Gesamtmasse gewogen. Die
Masse von Deckel und Faden wurden vorgegeben, woraus sich die Reinmasse
$m_{\mathup{k}}$ der Probe ergibt. Die Energiezufuhr zur Probe erfolgt über ein
wassergefülltes Becherglas auf einer Herdplatte. Die Probe wird dazu möglichst
weit in das Wasser gehängt. Wasser und Probe werden durch die Herdplatte auf ca.
\SI{100}{\celsius} erhitzt. Ist die Temperatur erreicht, wird die Temperatur des
Wassers $T_w$ im Dewar-Gefäß gemessen und der Probenkörper sofort in selbiges
gehängt. Durch eine schmale Öffnung im Deckel kann die Temperatur des Wassers
weiterhin über ein Digitalthermometer beobachtet werden. Stellt sich auch hier
nach einigen Minuten eine stationäre Temperatur ein, so ist dies die
Mischungstemperatur $T_{\mathup{m}}$. Die Messung wird für die Probe aus Blei
dreifach durchgeführt, für die Probe aus Graphit ebenfalls dreifach und für die
Probe aus Almuminium lediglich einmal.
\section{Auswertung}
\label{sec:auswertung}
\subsection{Fehlerrechnung}
Die in der Auswertung verwendeten Mittelwerte mehrfach gemessener Größen sind gemäß der
Gleichung
\begin{equation}
    \bar{x}=\frac{1}{n}\sum_{i=1}^n x_i
    \label{eqn:mittelwert}
\end{equation}
bestimmt. Der Standardfehler des Mittelwertes ergibt sich dabei zu

\begin{equation}
    \mathup{\Delta}\bar{x}=\sqrt{\frac{1}{n(n-1)}\sum_{i=1}^n\left(x_i-\bar{x}\right)^2}.
    \label{eqn:stdfehler}
\end{equation}
Resultiert eine Größe über eine Gleichung aus mehreren anderen fehlerbehafteten Größen, so
berechnet sich der Gesamtfehler nach der Gaußschen Fehlerfortpflanzung zu

\begin{equation}
    \mathup{\Delta}f(x_1,x_2,...,x_n)=\sqrt{\left(\frac{\mathup{d}f}{\mathup{d}x_1}\mathup{\Delta}x_1\right)^2+\left(\frac{\mathup{d}f}{\mathup{d}x_2}\mathup{\Delta}x_2\right)^2+ \dotsb +\left(\frac{\mathup{d}f}{\mathup{d}x_n}\mathup{\Delta}x_n\right)^2}.
    \label{eqn:gauß}
\end{equation}

\subsection{Messwerte und verwendete Literaturwerte}
\begin{table}
  \centering
  \caption{Thermospannungen und Masse des Wassers (Blei(Pb))}
  \label{tab:ThUmW}
  \sisetup{table-format=1.3}
  \begin{tabular}{
    S[table-format=3.2]
    S
    @{${}\pm{}$}
    S
    S[table-format=1.1]
    @{${}\pm{}$}
    S[table-format=1.1]
    S
    @{${}\pm{}$}
    S
    }
    \toprule{$m_w \,/\, \si{\gram}$} &
    \multicolumn{2}{c}{$U_w \,/\, \si{\milli\volt}$} &
    \multicolumn{2}{c}{$U_k \,/\, \si{\milli\volt}$} &
    \multicolumn{2}{c}{$U_m \,/\, \si{\milli\volt}$} \\
    \midrule
    593.97 & 0.914 & 0.001 & 2.6 & 0.2 & 0.965 & 0.005 \\
    575.55 & 0.89 & 0.01 & 2.9 & 0.3 & 0.915 & 0.005 \\
    566.31 & 0.877 & 0.001 & 3.0 & 0.5 & 0.924 & 0.001 \\
    \bottomrule
  \end{tabular}
\end{table}

\begin{table}
  \centering
  \caption{Thermospannungen und Masse des Wassers (Graphit(C))}
  \label{tab:ThUmW}
  \sisetup{table-format=1.3}
  \begin{tabular}{
    S[table-format=3.2]
    S
    @{${}\pm{}$}
    S
    S[table-format=1.1]
    @{${}\pm{}$}
    S[table-format=1.1]
    S
    @{${}\pm{}$}
    S
    }
    \toprule{$m_w \,/\, \si{\gram}$} &
    \multicolumn{2}{c}{$U_w \,/\, \si{\milli\volt}$} &
    \multicolumn{2}{c}{$U_k \,/\, \si{\milli\volt}$} &
    \multicolumn{2}{c}{$U_m \,/\, \si{\milli\volt}$} \\
    \midrule
    580.33 & 0.858 & 0.001 & 3.8 & 0.1 & 1.077 & 0.001 \\
    592.24 & 0.856 & 0.001 & 3.8 & 0.1 & 0.955 & 0.001 \\
    578.50 & 0.868 & 0.001 & 3.9 & 0.1 & 0.974 & 0.001 \\
    \bottomrule
  \end{tabular}
\end{table}

\begin{table}
  \centering
  \caption{Thermospannungen und Masse des Wassers (Aluminium(Al))}
  \label{tab:ThUmW}
  \sisetup{table-format=1.3}
  \begin{tabular}{
    S[table-format=3.2]
    S
    @{${}\pm{}$}
    S
    S[table-format=1.1]
    @{${}\pm{}$}
    S[table-format=1.1]
    S
    @{${}\pm{}$}
    S
    }
    \toprule{$m_w \,/\, \si{\gram}$} &
    \multicolumn{2}{c}{$U_w \,/\, \si{\milli\volt}$} &
    \multicolumn{2}{c}{$U_k \,/\, \si{\milli\volt}$} &
    \multicolumn{2}{c}{$U_m \,/\, \si{\milli\volt}$} \\
    \midrule
    580.05 & 0.868 & 0.001 & 3.8 & 0.1 & 0.988 & 0.001 \\
    \bottomrule
  \end{tabular}
\end{table}
\clearpage
Für die Massen der drei Körper wurden die folgenden Werte gemessen:
\begin{align}
  &m_{\mathup{Pb}} = \SI{385.76}{\gram} \\
  &m_{\mathup{C}} = \SI{114.55}{\gram} \\
  &m_{\mathup{Al}} = \SI{108.72}{\gram}
\end{align}
Die für die Bestimmung von $c_gm_g$ benötigten Massen des Wassers sowie die entsprechenden Thermospannungen betragen:
\begin{align}
  &m_x = \SI{346.27}{\gram} & &U_x = \SI{0.859(3)}{\milli\volt} \\
  &m_y = \SI{271.67}{\gram} & &U_y = \SI{4.10(1)}{\milli\volt}\\
  &m_m = \SI{617.94}{\gram} & &U'_m = \SI{2.150(5)}{\milli\volt}
\end{align}
Um eine lineare Näherung für die Abhängigkeit der Thermospannung von der Temperatur zu erhalten
wurden die Thermospannungen bei $\SI{0}{\celsius}$ und $\SI{100}{\celsius}$ gemessen.
\begin{align}
  &U_{\SI{0}{\celsius}} = \SI{-0.002(1)}{\milli\volt} \\
  &U_{\SI{100}{\celsius}} = \SI{4.10(1)}{\milli\volt}
\end{align}
Die gegebene Wärmekapazität von Wasser beträgt $c_w = \SI{4.18}{\joule\per\gram\per\kelvin}$.
Für weitere Rechnungen werden die folgenden Daten der verschiedenen Stoffe verwendet:
\begin{align}
  &M_{\mathup{Pb}} = \SI{207.2}{\gram\per\mol} & &\rho _{\mathup{Pb}} = \SI{11.35}{\gram\per\cubic\centi\meter} \\
  &\alpha _{\mathup{Pb}} = \SI{29e-6}{\per\kelvin} & &k_{\mathup{Pb}} = \SI{42e9}{\newton\per\square\meter} \\
  &M_{\mathup{C}} = \SI{12}{\gram\per\mol} & &\rho _{\mathup{C}} = \SI{2.25}{\gram\per\cubic\centi\meter} \\
  &\alpha _{\mathup{C}} = \SI{8e-6}{\per\kelvin} & &k_{\mathup{C}} = \SI{33e9}{\newton\per\square\meter} \\
  &M_{\mathup{Al}} = \SI{27}{\gram\per\mol} & &\rho _{\mathup{Al}} = \SI{2.7}{\gram\per\cubic\centi\meter} \\
  &\alpha _{\mathup{Al}} = \SI{23.5e-6}{\per\kelvin} & &k_{\mathup{Al}} = \SI{75e9}{\newton\per\square\meter}
\end{align}
\subsection{Bestimmung von T(U)}
Zu Beginn wird angenommen, dass der Verlauf von $U \left( T \right)$ näherungsweise linear ist.
\begin{equation}
  U \left( T \right) = a \cdot T + U_{\SI{0}{\celsius}}
  \label{eqn:U(T)}
\end{equation}
Mit $U_{\SI{0}{\celsius}}$ und $U_{\SI{100}{\celsius}}$ erhalt man dann zunächst die Steigung a.\\
$a = \SI{0.04102(10)}{\milli\volt\per\celsius}$\\
Damit folgt für $T \left( U \right)$:
\begin{equation}
  T \left( U \right) = \frac{1}{a} \cdot \left( U - U_{\SI{0}{\celsius}} \right)
  \label{eqn:T(U)}
\end{equation}
mit $\frac{1}{a} = \SI{24.38(6)}{\celsius\per\milli\volt}$.
\clearpage
\subsection{Berechnung der Wärmekapazitäten sowie der Atomwärmen}
Zuerst bestimmt man $c_{\mathup{g}}m_{\mathup{g}}$ für das Kalorimeter.\\
Hierzu rechnet man die gemessenen Spannungen $U_x$, $U_y$ und $U'_m$ in Temperaturen um
und setzt diese dann, zusammen mit $c_w$, in \eqref{eqn:cgmg} ein.
\begin{table}
  \centering
  \caption{Ergebnisse}
  \label{tab:Ergebnissecgmg}
  \sisetup{table-format=2.2}
  \begin{tabular}{
    S
    @{${}\pm{}$}
    S
    S[table-format=3.2]
    @{${}\pm{}$}
    S
    S
    @{${}\pm{}$}
    S
    S[table-format=3.0]
    @{${}\pm{}$}
    S[table-format=2.0]
    }
    \toprule
    \multicolumn{2}{c}{$T_x \,/\, \si{\celsius}$} &
    \multicolumn{2}{c}{$T_y \,/\, \si{\celsius}$} &
    \multicolumn{2}{c}{$T'_m \,/\, \si{\celsius}$} &
  \multicolumn{2}{c}{$c_gm_g \,\,/\,\, \si{\joule\per\kelvin}$} \\
    \midrule
    20.99 & 0.09 & 100.00 & 0.34 & 52.46 & 0.18 & 268 & 15 \\
    \bottomrule
  \end{tabular}
\end{table}\\
Danach rechnet man die restlichen gemessenen Spannungen in Temperaturen um und setzt
diese, zusammen mit den jeweiligen Massen und $c_gm_g$, in \eqref{eqn:ck} ein.
\begin{table}
  \centering
  \caption{Temperaturen (Blei)}
  \label{tab:tempPb}
  \sisetup{table-format=2.2}
  \begin{tabular}{
    S
    @{${}\pm{}$}
    S
    S[table-format=2.0]
    @{${}\pm{}$}
    S[table-format=2.0]
    S
    @{${}\pm{}$}
    S
    }
    \toprule
    \multicolumn{2}{c}{$T_w \,/\, \si{\celsius}$} &
    \multicolumn{2}{c}{$T_k \,/\, \si{\celsius}$} &
    \multicolumn{2}{c}{$T_m \,/\, \si{\celsius}$} \\
    \midrule
    22.33 & 0.06 & 63 & 5 & 23.57 & 0.14 \\
    21.75 & 0.25 & 71 & 7 & 22.35 & 0.13 \\
    21.43 & 0.06 & 73 & 12 & 22.57 & 0.06 \\
    \bottomrule
  \end{tabular}
\end{table}
\begin{table}
  \centering
  \caption{Temperaturen (Graphit)}
  \label{tab:tempC}
  \sisetup{table-format=2.2}
  \begin{tabular}{
    S
    @{${}\pm{}$}
    S
    S[table-format=2.1]
    @{${}\pm{}$}
    S[table-format=1.1]
    S
    @{${}\pm{}$}
    S
    }
    \toprule
    \multicolumn{2}{c}{$T_w \,/\, \si{\celsius}$} &
    \multicolumn{2}{c}{$T_k \,/\, \si{\celsius}$} &
    \multicolumn{2}{c}{$T_m \,/\, \si{\celsius}$} \\
    \midrule
    20.97 & 0.06 & 92.7 & 2.4 & 26.30 & 0.07 \\
    20.92 & 0.06 & 92.7 & 2.4 & 23.33 & 0.06 \\
    21.21 & 0.06 & 95.1 & 2.4 & 23.79 & 0.07 \\
    \bottomrule
  \end{tabular}
\end{table}
\clearpage
\begin{table}
  \centering
  \caption{Temperaturen (Aluminium)}
  \label{tab:tempAl}
  \sisetup{table-format=2.2}
  \begin{tabular}{
    S
    @{${}\pm{}$}
    S
    S[table-format=2.1]
    @{${}\pm{}$}
    S[table-format=1.1]
    S
    @{${}\pm{}$}
    S
    }
    \toprule
    \multicolumn{2}{c}{$T_w \,/\, \si{\celsius}$} &
    \multicolumn{2}{c}{$T_k \,/\, \si{\celsius}$} &
    \multicolumn{2}{c}{$T_m \,/\, \si{\celsius}$} \\
    \midrule
    21.21 & 0.06 & 92.7 & 2.4 & 24.13 & 0.07 \\
    \bottomrule
  \end{tabular}
\end{table}
Damit erhält man dann die spezifischen Wärmekapazitäten der Proben.
\begin{table}
  \centering
  \caption{Spezifische Wärmekapazitäten der Proben
  (0.0 ist ein Lückenfüller für nicht vorhandene Messreihen)}
  \label{tab:sezWärme}
  \sisetup{table-format=1.2}
  \begin{tabular}{
    S
    @{${}\pm{}$}
    S
    S
    @{${}\pm{}$}
    S
    S
    @{${}\pm{}$}
    S
    }
    \toprule
    \multicolumn{2}{c}{$c_{\mathup{Pb}} \,\,/\,\, \si{\joule\per\gram\per\kelvin}$} &
    \multicolumn{2}{c}{$c_{\mathup{C}} \,\,/\,\, \si{\joule\per\gram\per\kelvin}$} &
    \multicolumn{2}{c}{$c_{\mathup{Al}} \,\,/\,\, \si{\joule\per\gram\per\kelvin}$} \\
    \midrule
    0.22 & 0.04 & 1.89 & 0.07 & 1.06 & 0.04 \\
    0.09 & 0.04 & 0.83 & 0.03 & 0.0 & 0.0 \\
    0.15 & 0.04 & 0.85 & 0.03 & 0.0 & 0.0 \\
    \bottomrule
  \end{tabular}
\end{table}\\
Als nächstes berechnet man die Atomwärmen $C_{\mathup{P}}$ und $C_{\mathup{V}}$.
Die entsprechenden Formeln lauten
\begin{equation}
  C_{\mathup{P}} = c_{\mathup{k}} \cdot M_{\mathup{Pb}}
  \label{eqn:Cp}
\end{equation}
und
\begin{equation}
  C_{\mathup{V}} = C_{\mathup{P}} - 9 \alpha^2 k V_0 T_m
  \label{eqn:CvBez19}
\end{equation}
wobei $V_0$ das Molvolumen $\frac{M}{\rho}$ des Probenmaterials ist.
Somit erhält man mit den Messwerten folgende Werte für $C_{\mathup{P}}$ und $C_{\mathup{V}}$:
\begin{table}
  \centering
  \caption{Werte für $C_{\mathup{P}}$}
  \label{tab:CP}
  \sisetup{table-format=2.1}
  \begin{tabular}{
    S
    @{${}\pm{}$}
    S
    S
    @{${}\pm{}$}
    S
    S
    @{${}\pm{}$}
    S
    }
    \toprule
    \multicolumn{2}{c}{$C_{\mathup{P}} \left( Pb \right) \,\,/\,\, \si{\joule\per\mol\per\kelvin}$} &
    \multicolumn{2}{c}{$C_{\mathup{P}} \left( C \right) \,\,/\,\, \si{\joule\per\mol\per\kelvin}$} &
    \multicolumn{2}{c}{$C_{\mathup{P}} \left( Al \right) \,\,/\,\, \si{\joule\per\mol\per\kelvin}$} \\
    \midrule
    46 & 7 & 22.7 & 0.9 & 28.5 & 1.1 \\
    18 & 9 & 10.0 & 0.4 & 0.0 & 0.0 \\
    32 & 8 & 10.2 & 0.4 & 0.0 & 0.0 \\
    \bottomrule
  \end{tabular}
\end{table}
\begin{table}
  \centering
  \caption{Werte für $C_{\mathup{V}}$}
  \label{tab:CV}
  \sisetup{table-format=2.1}
  \begin{tabular}{
    S
    @{${}\pm{}$}
    S
    S
    @{${}\pm{}$}
    S
    S
    @{${}\pm{}$}
    S
    }
    \toprule
    \multicolumn{2}{c}{$C_{\mathup{V}} \left( Pb \right) \,\,/\,\, \si{\joule\per\mol\per\kelvin}$} &
    \multicolumn{2}{c}{$C_{\mathup{V}} \left( C \right) \,\,/\,\, \si{\joule\per\mol\per\kelvin}$} &
    \multicolumn{2}{c}{$C_{\mathup{V}} \left( Al \right) \,\,/\,\, \si{\joule\per\mol\per\kelvin}$} \\
    \midrule
    46 & 7 & 22.7 & 0.9 & 28.4 & 1.1 \\
    18 & 9 & 10.0 & 0.4 & 0.0 & 0.0 \\
    32 & 8 & 10.2 & 0.4 & 0.0 & 0.0 \\
    \bottomrule
  \end{tabular}
\end{table}

\section{Diskussion}
\begin{table}
  \centering
  \caption{Literaturwerte für $c_{\mathup{k}}$}
  \label{tab:ckl}
  \sisetup{table-format=1.3}
  \begin{tabular}{
    S
    S
    S
    }
    \toprule
    {$c_{\mathup{Pb}} \,\,/\,\, \si{\joule\per\gram\per\kelvin}$} &
    {$c_{\mathup{C}} \,\,/\,\, \si{\joule\per\gram\per\kelvin}$} &
    {$c_{\mathup{Al}} \,\,/\,\, \si{\joule\per\gram\per\kelvin}$} \\
    \midrule
    0.129 & 0.715 & 0.896 \\
    \bottomrule
  \end{tabular}
\end{table}
Mit den Literaturwerten für $c_{\mathup{k}}$ ergeben sich dann die folgenden theoretischen Werte für $C_{\mathup{P}}$
und $C_{\mathup{V}}$:
\begin{table}
  \centering
  \caption{Literaturwerte für $C_{\mathup{P}}$}
  \label{tab:ckl}
  \sisetup{table-format=2.2}
  \begin{tabular}{S S S}
    \toprule
    {$C_{\mathup{P}} \left( Pb \right) \,\,/\,\, \si{\joule\per\mol\per\kelvin}$} &
    {$C_{\mathup{P}} \left( C \right) \,\,/\,\, \si{\joule\per\mol\per\kelvin}$} &
    {$C_{\mathup{P}} \left( Al \right) \,\,/\,\, \si{\joule\per\mol\per\kelvin}$} \\
    \midrule
    26.73 & 8.58 & 24.19 \\
    \bottomrule
  \end{tabular}
\end{table}
\clearpage
\begin{table}
  \centering
  \caption{Literaturwerte für $C_{\mathup{V}}$}
  \label{tab:ckl}
  \sisetup{table-format=2.2}
  \begin{tabular}{S S S}
    \toprule
    {$C_{\mathup{V}} \left( Pb \right) \,\,/\,\, \si{\joule\per\mol\per\kelvin}$} &
    {$C_{\mathup{V}} \left( C \right) \,\,/\,\, \si{\joule\per\mol\per\kelvin}$} &
    {$C_{\mathup{V}} \left( Al \right) \,\,/\,\, \si{\joule\per\mol\per\kelvin}$} \\
    \midrule
    26.60 & 8.58 & 24.10 \\
    \bottomrule
  \end{tabular}
\end{table}
Auffällig ist, dass die jeweils ersten Messreihen von Blei und Graphit stark aus der Reihe
tanzen und die dabei entstandenen Ergebnisse damit doppelt so groß sind, wie die, der
anderen beiden Messreihen. Für die erste Messung mit Blei würden wir unsere Unerfahrenheit
mit der Apparatur verantwortlich machen sowie der unhandliche Zustand der Probe, welche
es aufgrund einer nicht intakten Einkerbung für das Thermoelement die Messung wesentlich
erschwert hat. Für die Abweichung bei der ersten Messung für Graphit finden wir keinen konkreten Grund.
Allerdings ist uns im Laufe der verschiedenen Messungen aufgefallen, dass die Mischtemperatur teilweise
stark schwankt, und zwischenzeitlich einen scheinbar konstanten Wert annimmt und dann wieder auf
einen kleineren konstanten Wert abfällt.\\
Abgesehen davon liegen die Abweichungen der restlichen Werte für Blei im Rahmen der statistischen Fehler.
Für Graphit und Aluminium liegen die restlichen Ergebnisse ca. $\SI{0.1}{\joule\per\gram\per\kelvin}$ über den
Literaturwerten. Fehlerquellen dafür könnten z.B. die möglicherweise ungenaue Annahme eines
linearen Verlaufes für $U \left( T \right)$ sein.
Des Weiteren gestalltet sich die Temperaturmessung für $T_{\mathup{k}}$ als etas kompliziert,
da die Temperatur beim Tranfer der Probe zum Kalorimeter teilweise schon zügig abfällt.
Aufgrund des dem entsprechend schnelleren Vorgangs fällt diese Messung immer relativ ungenau aus.
Ein weiterer Faktor ist, dass zwischen der Messung der Temperatur des Wassers und der Messung
der Temperatur der Probe sowie der Mischtemperatur relativ viel Zeit vergeht, da man das
Thermoelement zu Beginn des Aufheizens der Probe an der Probe befestigt, um den Transfervorgang
zu vereinfachen. Dies sorgt dafür, dass äußere Einflüsse, die in dieser Zeit die Wassertemperatur
verändern, wie z.B. Abwärme von der Heizplatte, nicht berücksichtigt werden.\\
Insgesamt kann man gut erkennen, dass für schwere Elemente wie Blei der klassisch angenommene Wert
von 3R gut passt. Bei Graphit dagegen zeigt sich deutlich der durch die Quantendynamik beschriebene
Effekt, dass ein Stoff je nach molarer Masse fürher oder später den Wert von 3 R erreicht.
\clearpage
\nocite{*}
\printbibliography

\end{document}
